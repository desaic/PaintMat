Multi-material 3D printing allows objects to be composed of complex, heterogeneous arrangements of materials.
It is often more natural to define a functional goal than to define the material composition of an object.
Translating these functional requirements to fabricable 3D prints is still an open research problem.
Recently, several specific instances of this problem have been explored
(e.g.,  appearance or elastic deformation),
but they exist as isolated, monolithic algorithms.
In this research, I propose an abstraction mechanism that simplifies the design, development, implementation, and reuse of these algorithms.
The solution relies on two new data structures:
a \emph{reducer tree} that efficiently parameterizes the space of material assignments
and a \emph{tuner network} that describes the optimization process used to compute material arrangement.
I provide an application programming interface for specifying the desired object and for defining parameters
for the \emph{reducer tree} and \emph{tuner network}. I illustrate the utility of my new framework
by implementing several fabrication algorithms as well as demonstrating the manufactured results.
