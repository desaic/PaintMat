\chapter{Related Work}
\label{chap:relate}
% Geometry compression: Digital Bas-Relief from 3D Scenes
The new data structures draw ideas from previous work in rendering and optimization.
The \emph{reducer tree} is inspired by Cook's shade trees~\cite{Cook1984} and their modern implementation in current rendering systems (e.g., Maya, RenderMan, etc.). Using these approaches, complex effects can be achieved by combining a set of basic shading blocks. The \emph{reducer tree} also uses a tree data structure that combines a set of primitives to compute a material assignment for each point inside of an object volume and describes a spatial-partition of the object volume. While there are many space-partitioning data structures (BSPs, Quad-trees, KD-trees, etc.), they are difficult to specify by hand and they are typically tied to the object geometry.
My \emph{reducer tree} is intuitive to construct but sufficiently general for representing material distributions.
In addition, it is specified in an object-independent manner -- the same \emph{reducer tree} can be reused for processing different objects.
My second, new data structure which is responsible for optimizing material assignment is the \emph{tuner network}.
It is inspired by probabilistic graphical models~\cite{Jordan:1999} that have been popular for representing variable dependence in multi-variate optimization problems.
These problems can be parallelized~\cite{GraphLab} according to the associated graph structure.

This work is also inspired by the many instances of specification to fabrication translation pipelines that have been proposed recently.
These pipelines span a wide range of functional goals (e.g., mechanical, optical, appearance).
They drive the simulation, optimization, and reductions chosen for each method. 
One example of such processes is recent work on optimizing material composition
to achieve prescribed deformation behavior. 
Bickel and colleagues~\shortcite{Bickel:2010:DAF} have designed a system
for manufacturing multi-layer composites with a given elastic behavior.
Skouras \textit{et al}.~\shortcite{sko:2012} provide design and construction of balloons
with a prescribed shape while inflated.
Furthermore, material optimization has been employed to create mechanical clones of human faces ~\cite{Bickel:2012}.

Optimizing and manufacturing objects with desired appearance and optical characteristics has also been explored. Weyrich \textit{et al}.~\shortcite{Weyrich:2009:FMF} compute surface micro-geometry that yields a desired BRDF. Similar approaches have been proposed~\cite{Finckh:2010,Marios:2011} to produce refractive surfaces that form user-defined caustics. These methods have been extended to optically decrypt hidden images~\cite{Papas:2012}. Complementary work examines fabricating surfaces with spatially varying reflectance~\cite{Matusik:2009:PSR,Malzbender:2012:PRF} and diffuse shading \cite{Alexa:2010:RAI}. Another set of approaches uses optimization to compute shadow casting surfaces and volumes~\cite{Mitra:2009:SA,Bermano:2012,Baran:2012:MLA} that reproduce a given set of input images. Finally, optimization-based approaches have also been employed to control the subsurface scattering of 3D printed multi-layered models~\cite{Dong:2010:FSS,Hasan:2010:PRO}. I seek to exploit the common form of the above works in order to generalize them. 

There have been studies of material assignment representations in other fields, primarily in mechanical engineering. 
Here I only list a few representative works.
Kumar \textit{et al}.~\shortcite{Kumar1999} describe material composition by dividing a volume into sub-volumes.
They perform material interpolation  using local, sub-volume coordinate systems. 
Jackson ~\shortcite{Jackson2000} explores several mesh data structures for spatial sub-division.
Kou and Tan~\shortcite{Kou2007} give a comprehensive review
of spatial partition schemes and material interpolation functions. 
Kou \textit{et al}.~\shortcite{Kou2012} use a hierarchy of procedures to define material composition with a small number of design parameters.
They run particle swarm optimization ~\cite{kennedy1995}
on the design parameters to minimize thermal stress of an object.
Their work focused on smoothly varying materials.
There is no discussion of high-frequency discrete material assignment for capturing details.
They are limited to global optimization algorithms
such as Particle Swarm Optimization because the dependency structure between the design variables is not modeled.
In computer graphics, procedural descriptions for material assignment have also been studied.
Cutler \textit{et al}.~\shortcite{Cutler2002} describe a scripting language for specifying layered solid models and describing material composition.
Vidim\v{c}e \textit{et al}.~\shortcite{Vidimce2013} provide a fabrication language and a programmable pipeline for specifying material composition directly and precisely throughout a volume.
In contrast, Spec2Fab can be used to design algorithms that only require object properties
rather than a precise description of material assignment.

In this research I focus on constructing a common model for encompassing the entire Spec2Fab problem.
I am motivated by the similarities present in state-of-the-art algorithms.
\autoref{tb:algorithms} shows the coordinate reduction methods as well as the optimization schemes
used in all these prior approaches.
A key observation is that previous works share a common methodology and  a common set reduced coordinate components. These components are then combined with (often off-the-shelf) optimization techniques
to form new material assignment algorithms.
This suggests that one could find a small set of components that can be tied together to synthesize  advanced methods.
I attempt to identify these components as well as a suitable model for their interaction.

\begin{table*}[htp]
\centering
\begin{tabular}{clll}
\hline
\textbf{Paper} & \textbf{Goal} & \textbf{Reduced Coordinates}  & \textbf{Optimization} \\
\hline
~\cite{Finckh:2010} & Optical (Caustics) & B-Spline Surface & Stochastic Approximation \\
~\cite{Marios:2011} & Optical (Caustics) & Piecewise Constant Tiles & Simulated Annealing \\
~\cite{Alexa:2010:RAI}& Optical (Relief) & Height Field & Gradient Descent\\
~\cite{Weyrich:2009:FMF} &Optical (Reflectance)&Piecewise Constant Tiles& Simulated Annealing\\
~\cite{Papas:2012} & Optical (Refraction) & Piecewise Constant Tiles & Simulated Annealing \\
~\cite{Mitra:2009:SA}&Optical (Shadows)&Voxel Grid&Custom Discrete\\
~\cite{Baran:2012:MLA}&Optical (Shadows)& Layered Materials & Quadratic Program \\
~\cite{Bermano:2012} & Optical (Shadows) & Height Field & Custom/Simulated Annealing \\
~\cite{Dong:2010:FSS}&Optical (Subsurface) &Layered Materials&Conjugate Gradient\\
~\cite{Hasan:2010:PRO}& Optical (Subsurface) & Layered Materials & Branch and Bound\\
~\cite{Bickel:2010:DAF}& Mechanical (force) & Layered Materials  & Branch and Bound \\
~\cite{Bickel:2012} & Mechanical (shape) &  Height Field &  Newton-Raphson \\
~\cite{sko:2012}& Mechanical (shape) & Triangle Mesh & Augmented Lagrangian Method \\
\hline
\end{tabular}
\caption{The goal type, reduction type and optimization used by prior computational fabrication approaches.}
\label{tb:algorithms}
\end{table*}