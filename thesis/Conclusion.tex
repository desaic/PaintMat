\chapter{Conclusion}
In this research I have taken the first step towards solving an open problem in computational fabrication -- 
creating a general translation process that transforms user-defined model specifications into printer and material-specific representations.
My process relies on two data structures to make this general translation process expressive and computationally tractable: a \emph{reducer tree} and a \emph{tuner network}.
I will show how existing instances of this translation can be expressed and combined within my system.
I believe that my API and its reference implementation will simplify and encourage development of new translation processes.

My system can offer many exciting opportunities for future work.
First, it would be extremely useful to implement many additional simulators in order to allow computing a variety of other properties, e.g., structural soundness, stability, material cost, and printing time.
These simulators could be employed to expand the range of possible user-defined specifications.
Similarly, only relatively simple error metrics have been proposed within the tuning process.
The development of more sophisticated and, in particular, perceptually-driven material metrics remains a relatively unexplored research area.
Finally, it would be very beneficial to couple my API with a visual interface to further simplify the task of translator construction.
It is not obvious what the best visual interface for specifying functional or physical properties of these objects is and more research in this area is necessary.
